\documentclass{article}

\usepackage{csvsimple}
\usepackage[numbers]{natbib}
\usepackage[utf8]{inputenc}

\usepackage[a4paper,width=150mm,top=25mm,bottom=25mm]{geometry}
\usepackage{fancyhdr}
\pagestyle{fancy}
\fancyhf{}
\fancyfoot[R]{\thepage}
\renewcommand{\headrulewidth}{0pt}
 \renewcommand{\footrulewidth}{0pt}


\usepackage[export]{adjustbox}



\begin{document}

\begin{figure}[h]
  \centerline{\small MAKERERE 
  \includegraphics[width=0.2\textwidth]  {mak-logo-sm.png} UNIVERSITY\\}
 \end{figure}

\centerline{COLLEGE OF COMPUTING AND INFORMATION SCIENCES\\}
\centerline{DEPARTMENT OF COMPUTER SCIENCE\\}
\centerline{COURSEWORK: RESEARCH METHODOLOGY(BIT 2207)\\}
\centerline{LECTURER: ERNERST MWEBAZE\\}


\begin{titlepage}

	\begin{center}
	\line(1,0){300}\\

	\huge{\bfseries  Literature Review: Mobile Application Development using Android Studio}\\
	[2mm]
	\line(1,0){200}\\
	[1.5cm]

	\end{center}

	\begin{flushright}
	
	\textsc{\large Nabimanya Lynn \\}
	\# 216008283 \\
	

	\end{flushright}
\end{titlepage}


\cleardoublepage

\section{INTRODUCTION}\label{sec:intro}
Android Studio is an platform developed to facilitate and development of mobile applications
When starting a mobile development, few questions may be raised. These questions are answered as each literature is reviewed in the following sections 
The following papers and a book have been chosen to be reviewed in each section: 
Paper 1 - Software Engineering Issues for Mobile Application Development 
Paper 2 - Mobile Application Development: Web vs. Native\cite{citation01}
Paper 3 - Smart Smartphone Development: iOS versus Android \cite{citation02} 
Book - Mobile Interaction Design \cite{citation03}


\section{2. Software Engineering Issues in Mobile Development }
Paper 1 - Software Engineering Issues for Mobile Application Development 
Firstly, was the potential interaction of applications between each other. Secondly, the sensor handling was pointed out. The accelerometers that respond to device movements, numerous touch screen gestures, global positioning system, cameras, and multiple networking protocols are all in a single device. Lastly, the power consumption and battery life was brought up. 
\section{3. Web vs. Native Application}
The article started with the argument “developers cannot develop for every platform”. One of the disadvantages for Web applications was the user interface code. The authors stated that the standard APIs for Web application interfaces are much weaker than the native applications. User experience was another area to have an effect on both native and web application development as the Web applications must be connected to the Internet the entire time the application is running but native applications can work offline as well as online.
\section{4. iOS and Android Operating System }
Paper 3 - Smart Smartphone Development: iOS versus Android [citation04] 
Today, there are at least five important platforms (iPhone, Android, Blackberry, Windows Phone, Symbian) [citation04], but a large portion of the mobile markets in the world are currently iPhone and Android 
\section{5. Human-Computer Interaction in Mobile }
The book was reviewed for general human-computer interaction (HCI) and how HCI is different for mobile application. 
Book - Mobile Interaction Design \cite{citation04} \\
The main emphasis of the book was on the user satisfaction of the design and functionality. The authors argued that even if the industrial design and their aesthetic are appealing, if it does not address real user needs, it will be no use to the user. 
\section{6.Conclusion}
Four areas that can be helpful in starting mobile development for have been reviewed. Couple of issues were raised in the software engineering section without solutions. These issues should be considered for complex applications and there need to be further research in the future on the issues that were raised. In deciding whether to develop a native application or web application, the developer would have to consider user experience as well as time and cost constraints to make the best choice. The HCI section provided insight in designing user-centered applications. User requirements and functionality should be the focus in the application design.  
\cleardoublepage

\begin{thebibliography}{9}


\bibitem{citation01}
 Charland, A., and Leroux, B. 2011. Mobile Application Development: Web vs. Native,
  \textit
  Communications of the ACM. ACM,
  v.54. n.5. DOI=10.1145/1941487.1941504.

\bibitem{citation02}  
Goadrich, M. H., and Rogers, M. P. 2011. Smart Smartphone Development: iOS versus Android,
  \textit
  Goadrich, M. H., and Rogers, M. P. 2011. Smart Smartphone Development: iOS versus Android. ACM,
  September 16 2009.


\bibitem{citation03}
 Jones, M and Marsden G,
  \textit
	Mobile Interaction Design John Wiley and Sons Ltd,
	2006.

\bibitem{citation04}
Padley  R,
  \textit
	HTML5 - Bridging the Mobile Platform Gap: Mobile Technologies in Scholarly Communication. Serials: The Journal for the Serials 			Community (0953-0460). UKSG, Volume 24, Supplement 3, S32-S39,
	DOI=10.1629/24S32.


\end{thebibliography}


\bibliographystyle{IEEEtranN}
\bibliographystyle{References}

\end{document}
